\newpage
\chapter*{Sommario}

\addcontentsline{toc}{chapter}{Sommario}
Questo lavoro di tesi fornisce un insieme di strumenti e metodologie di supporto alla segmentazione di mercato, cio� a quel processo di marketing che suddivide un variegato insieme di clienti in gruppi omogenei (o \textit{segmenti}) caratterizzati da simili comportamenti d'acquisto. Con l'avvento dei Big Data in generale e la pervasivit� dei Social Network in particolare, si � assistito ad una esplosione della complessit� delle basi di dati aziendali, che possono essere analizzate efficacemente solo con l'ausilio di moderne tecniche mutuate dalla cluster analysis. Nei prossimi capitoli descriveremo nel dettaglio lo stato dell'arte della analisi dei dati, individuando gli strumenti di ultima generazione che sono stati indispensabili per raggiungere gli obiettivi del lavoro: dalla raccolta, analisi e preparazione dei dati fino all'individuazione di innovativi algoritmi di clustering che operano sia su dataset classici che su complessi grafi con attributi; successivamente mostreremo come abbiamo implementato, integrato ed utilizzato ognuna di queste tecniche, evidenziandone gli aspetti peculiari e i punti di forza e di debolezza. Infine, presenteremo delle linee guida per suggerire le migliori parametrizzazioni dei vari algoritmi ed elevare, di conseguenza, la qualit� dei risultati.

\chapter*{Abstract}
\addcontentsline{toc}{chapter}{Abstract}
This thesis provides a set of tools and methodologies to support market segmentation, that is the marketing process that divides a diverse set of customers into homogeneous groups (or segments) characterized by similar buying behavior. With the advent of Big Data in general and the pervasiveness of social networks in particular, there has been an explosion of the complexity of corporate databases, which can be effectively analyzed only with the help of modern techniques borrowed from the field of cluster analysis. This work describes in detail the state of the art of data analysis, identifying the tools of last generation that have been essential in achieving the main goals: from the collection, study and preparation of the data up to the identification of innovative clustering algorithms that operate on both classic datasets and complex attributed graphs; then, we will show how each of these techniques was implemented, integrated and used, highlighting its peculiar aspects, strengths and weaknesses. Finally, we will present the guidelines to suggest the best parameter settings of the various algorithms and thus enhance the quality of the clustering results.