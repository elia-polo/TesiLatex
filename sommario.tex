\newpage
\chapter*{Sommario}

\addcontentsline{toc}{chapter}{Sommario}
Questo lavoro di tesi fornisce un insieme di strumenti e metodologie di supporto alla segmentazione di mercato, cio� a quel processo di marketing che suddivide un variegato insieme di clienti in gruppi omogenei (o \textit{segmenti}) caratterizzati da simili comportamenti d'acquisto. Con l'avvento dei Big Data in generale e la pervasivit� dei Social Network in particolare, si � assistito ad una esplosione della complessit� delle basi di dati aziendali, che possono essere analizzate efficacemente solo con l'ausilio di moderne tecniche mutuate dalla cluster analysis. Nei prossimi capitoli descriveremo nel dettaglio lo stato dell'arte individuando gli strumenti di ultima generazione che sono stati indispensabili per raggiungere gli obiettivi del lavoro: dalla raccolta, analisi e preparazione dei dati fino all'individuazione di innovativi algoritmi di clustering che operano sia su dataset classici che su complessi grafi con attributi; successivamente mostreremo come abbiamo implementato, integrato ed utilizzato ognuna di queste tecniche, evidenziandone gli aspetti peculiari e i punti di forza e di debolezza. Infine, presenteremo delle linee guida per suggerire le migliori parametrizzazioni dei vari algoritmi ed elevare, di conseguenza, la qualit� dei risultati.