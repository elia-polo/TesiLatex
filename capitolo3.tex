\chapter{La piattaforma Neosperience Engage}
\label{capitolo3}
\thispagestyle{empty}

Neosperience � una azienda italiana che offre servizi di marketing e \textit{customer experience}. Customer experience � la percezione che il cliente matura---razionalmente e emotivamente---della relazione con un marchio, un fornitore di beni o servizi. In un mercato globale \mbox{iper-competitivo} e \mbox{iper-connesso}, per contrastare la crescente dispersione degli utenti, il marchio deve stabilire un legame individuale con il cliente. La gestione dell'esperienza cliente \`e la pianificazione e la reazione alle interazioni del cliente al fine di raggiungerne ed eccederne le aspettative ed in questo modo aumentare la soddisfazione, la lealt\`a e il sostegno del cliente al marchio\footnote{``The practice of designing and reacting to customer interactions to meet or exceed customer expectations and, thus, increase customer satisfaction, loyalty and advocacy'' - Gartner}: in una parola, coinvolgimento o \textit{engagement}. L'esperienza cliente digitale apre nuove vie di contatto con il cliente: non solo portali internet, ma applicazioni per social network e dispositivi mobili, tecnologie 3D e negozi virtuali, realt� aumentata, \textit{gamification}. Il ciclo vitale di un cliente rispetto ad un marchio inizia con il coinvolgimento, la scoperta di un prodotto o un servizio; dopo questo primo contatto, seguono diversi \textit{moment of truth}, le occasioni in cui il cliente interagisce con l'azienda e si forma una opinione, consapevole ed inconscia. Il mantenimento del cliente dipende dalla capacit\`a di influire positivamente in ciascuno di questi istanti: la valutazione del prodotto, la decisione dell'acquisto e l'esperienza d'uso. Questo non \`e solo qualit\`a del servizio, ma arrivare ad una comprensione cos\`i profonda del cliente da poter offrire una esperienza---contenuti e benefici---tanto personalizzata ed appagante da indurlo non solo a restare leale al marchio ma a convincere altri ad avvicinarsi. Per raggiungere questo grado di conoscenza \`e necessario estrarre indizi da ogni punto di contatto con il cliente, sfruttando l'immensa mole di informazione nella rete.\\
Recentemente, Neosperience ha sviluppato Engage, una collezione gratuita di librerie per presentare ad un utente il miglior ordinamento degli oggetti di un catalogo, secondo un criterio di pertinenza, basandosi sul profilo dell'utente delineato dalla sua impronta sui social network.\\
Engage offre una interfaccia (API) o uno strumento di sviluppo (SDK) per arricchire o creare applicazioni per dispositivi mobili. Una applicazione costruita con Engage � una galleria di contenuti; ciascun prodotto del catalogo o \textit{deck} � presentato individualmente su una \textit{card}, un volantino digitale, che racchiude una intestazione, una immagine e una breve descrizione testuale. Attraverso le API Engage � inoltre facilmente possibile offrire funzionalit� di \mbox{e-commerce} e condivisione della card su social network. Ad ogni oggetto nella vetrina deve essere infine associato un \textit{target} o utente ideale, colui che l'autore dei contenuti ritiene essere pi� interessato all'oggetto o propenso ad acquistarlo. Il \textit{target} � definito mediante variabili demografiche---et�, genere, luogo di nascita, educazione---, interessi, avversioni e geolocazione---la distanza dell'utente da un luogo specificato. Quando l'utente installa e avvia l'applicazione pu� connettere i propri profili sui principali social network---Facebook Twitter e Foursquare---permettendo alla piattaforma Engage sottostante di tratteggiare un profilo dell'utente avvalendosi dei dati personali, dei \textit{like} e \textit{hashtag} sia dell'utente sia degli amici sul social network. Partendo da questo profilo, Engage misura per ogni oggetto nel catalogo dell'applicazione la somiglianza tra il \textit{target} dell'oggetto e l'utente; gli oggetti sono quindi riordinati e visualizzati dal pi� appropriato, quello per cui l'utente � massimamente simile al \textit{target}, al pi� distante.\\
Questo lavoro � stato motivato dall'esigenza di Neosperience di valorizzare i dati degli utenti raccolti tramite la propria piattaforma Engage. Questi dati difatti racchiudono il potenziale per un diverso approccio alla segmentazione della clientela, facendo leva su informazioni variegate, semi-strutturate e non strutturate, ma nuove e complementari rispetto alle tradizionali categorie sociali e geografiche. L'opportunit� offerta da questi dati � la conoscenza puntuale degli interessi, delle opinioni, del mood di ciascun cliente, e poter rispondere in tempo reale, limitati soltanto dalla capacit� di elaborare e assimilare questa informazione. Tuttavia, complice la giovinezza della piattaforma, la quantit� di dati a disposizione di Neosperience era molto al di sotto del volume prospettato a regime, sul quale gli algoritmi e le tecniche di data mining avrebbero dovuto essere messe alla prova. Pertanto, il lavoro si � articolato nei seguenti passi:
\begin{itemize}
\item \hyperref[studio_letteratura]{studio della letteratura}
\item \hyperref[raccolta_dati]{raccolta dei dati}
\item \hyperref[selezione_algoritmi]{selezione degli algoritmi}
\item \hyperref[preparazione_dati]{preparazione dei dati}
\item \hyperref[esecuzione_algoritmi]{esecuzione degli algoritmi}
\item \hyperref[analisi_risultati]{analisi dei risultati}
\end{itemize}
\section{Studio della letteratura}
\label{studio_letteratura}
\section{Raccolta dei dati}
\label{raccolta_dati}
Nel progetto originale, il framework � alimentato da tre sorgenti di dati: Facebook, Twitter e Foursquare.\\
Facebook, con oltre un miliardo di utenti mensili, � il pi� diffuso e noto social network al mondo. In Facebook, un utente pu� pubblicare la propria storia personale, interessi, esperienze, foto, lavoro e perfino stati d'animo. D'altronde, il fulcro dei social network � tessere relazioni, e Facebook non fa eccezione: ogni utente pu� stringere amicizia con altri utenti, condividere con essi contenuti, conversare, giocare, organizzare eventi ed essere informato di ogni cambiamento nella propria rete sociale.\\
\mbox{GRAPH API}\footnote{\url{https://developers.facebook.com/docs/graph-api}} � lo strumento primario per leggere e scrivere sul grafo sociale di Facebook. L'interfaccia offre una rappresentazione della enorme base dati Facebook sotto forma di grafo, composto da:
\begin{description}
\item[nodi]  basilarmente ogni entit�: un utente, una foto, una pagina, un commento
\item[archi] le connessioni tra entit�, ad esempio la foto inserita in una pagina o il commento allegato alla foto
\item[campi] informazioni sulle entit�, come il compleanno di un utente o il titolo di una pagina
\end{description}

(In questa parte Nicola � certamente pi� ferrato)
\\
Twitter � un social network e una piattaforma di microblogging. Al cuore di Twitter vi sono i \textit{tweet}, brevi messaggi in 140 caratteri che un utente pubblica e la piattaforma notifica a tutti i suoi contatti, chiamati \textit{follower}. Un utente riceve i tweet delle persone che segue sulla propria bacheca o \textit{home timeline}, da cui pu� rispondere, inserendosi nel flusso di tweet esistente, o ripubblicare (\textit{retweet}), propagando il tweet ai propri follower. Una evidente differenza fra Twitter e Facebook � proprio la natura e lo scopo delle interazioni tra gli utenti. Facebook � solitamente usato per stare in contatto o ritrovare persone che si conoscono realmente o si sono frequentate nel passato. In particolare, una volta stretta amicizia il rapporto tra due amici � paritario e simmetrico: possono scambiarsi liberamente messaggi ed ogni contenuto pubblicato da una parte viene notificato all'altra e viceversa. Al contrario, Twitter � principalmente adoperato per comunicare con altre persone con le quali, anche se non le frequentiamo nella vita reale, condividiamo interessi o argomenti di discussione. Twitter realizza un modello asimmetrico di relazione, in cui la discrepanza tra il numero di persone ci seguono e quelle che seguiamo determina la nostra reputazione all'interno della comunit�. Nella pratica, l'asimmetria limita i privilegi di colui che segue: ad esempio, il \textit{follower} non pu� contattare unilateralmente le persone che segue, a meno che non sia stato esplicitamente autorizzato o la relazione sia stata ricambiata. Questa peculiarit� ha sensibili implicazioni sul modello dei dati, giacch� il grafo degli utenti deve adesso contenere archi orientati. Inoltre, il significato e il peso degli archi introdotti da Twitter � radicalmente diverso da quello dettato da Facebook. La relazione \\
Per organizzare tematicamente le conversazioni, i tweet sono spesso etichettati con un \textit{hashtag}, una parola chiave preceduta da un cancelletto, o \textit{hash} in inglese. Questa convenzione nacque spontaneamente tra gli utenti di Twitter e fu raccolta ed integrata nella piattaforma, che oggi pubblica in tempo reale la lista dei \textit{trending topics}, le parole o argomenti di discussione che compaiono pi� frequentemente negli ultimi tweet.

\section{Selezione degli algoritmi}
\label{selezione_algoritmi}
\section{Preparazione dei dati}
\label{preparazione_dati}
\section{Esecuzione degli algoritmi}
\label{esecuzione_algoritmi}
\section{Analisi dei risultati}
\label{analisi_risultati}